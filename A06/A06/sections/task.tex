%!TEX root=../document.tex

\section{Einführung}

\subsection{Grundanforderungen}

\begin{itemize}
	\item Eine eigene Klasse erbt von Thread
	\item Die Klasse definiert eine gemeinsame Lock sowie einen gemeinsamen Counter
	\item Im Konstruktor wird über einen Parameter bestimmt, für welche Zahlen dieser Thread zuständig ist
	\item In der run-Methode wird die Summe korrekt aufsummiert, wobei der Zugriff auf den Counter über die Lock threadsicher gestaltet wird (with-Statement)
	\item Kommentare und Sphinx-Dokumentation
	\item Kurzes Protokoll über deine Vorgangsweise, Aufwand, Resultate, Beobachtungen, Schwierigkeiten, ... (Kopf- und Fußzeile etc.)
\end{itemize}

\subsection{Erweiterungen}

\begin{itemize}
	\item Miss die Laufzeit!
	\item Untersuche, wie sich die Laufzeit auf deinem System verhält, wenn du es mit mehr oder weniger Threads verwendest, z.B. Single Threaded (d.h. nur im main-Thread), mit 2 Threads, mit 3 Threads, ...
	\item Interpretiere die Ergebnisse und halte deine Erkenntnisse im Protokoll fest! Warum verhält es sich so?
	\item Finde eine Möglichkeit, wie die Performance verbessert werden kann und eventuelle Beschränkungen umgangen werden können!
\end{itemize}

\subsection{Voraussetzungen}

\begin{itemize}
	\item Python Kentnisse
	\item threading Kentnisse
\end{itemize}


\subsection{Aufgabenstellung}

Schreibe ein Programm, welches die Summe von 1 bis zu einer von dem/der Benutzer/in einzugebenden (potentiell sehr großen) Zahl mithilfe von drei Threads berechnet!
\clearpage
