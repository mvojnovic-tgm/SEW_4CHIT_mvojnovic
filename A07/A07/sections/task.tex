%!TEX root=../document.tex

\section{Einführung}

\subsection{Grundanforderungen}

\begin{itemize}
	\item Zwei eigene Klassen (Consumer und Producer) erben von Thread (E1 und V1)
	\item Die zwei Klassen sind über einen Queue verbunden
	\item Der Erzeuger E1 sucht nach Primzahlen. Jede gefundene Primzahl wird über die Queue an den Verbraucher V1 geschickt
	\item Der Verbraucher gibt die empfangene Zahl in der Konsole aus und schreibt sie außerdem in eine simple Textdatei
	\item Erzeuger und Verbraucher stimmen sich über Queue.task done() und Queue.join() ab Kommentare und Sphinx-Dokumentation
	\item Kurzes Protokoll über deine Vorgangsweise, Aufwand, Resultate, Beobachtungen, Schwierigkeiten, ... Bitte sauberes Dokument erstellen! (Kopf- und Fußzeile etc.)
\end{itemize}

\subsection{Erweiterungen}

\begin{itemize}
	\item Ein weiterer Thread nimmt Benutzereingaben entgegen
	\item Dieser Thread kann als ein weiterer Erzeuger E2 gesehen werden
	\item Wird eine (potentiell sehr große) Zahl eingegeben, so wird in einem weiteren Verbraucher V2 überprüft, ob es sich bei dieser produzierten Zahl um eine Primzahl handelt
	\item E2 und V2 müssen sich nicht über task done() absprechen, d.h. E2 kann mehrere Aufträge in die Queue schicken, bevor V2 mit der Bearbeitung fertig ist
	\item Wird "exit" eingegeben, so werden alle Threads sauber beendet
	\item Achte auf Fehlerfälle!
\end{itemize}

\subsection{Voraussetzungen}

\begin{itemize}
	\item Python Kentnisse
	\item threading Kentnisse
\end{itemize}


\subsection{Aufgabenstellung}

Schreibe ein Programm, welches ein simples Erzeuger-Verbraucher-Muster implementiert!
\clearpage
